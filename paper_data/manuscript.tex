\documentclass{elsarticle}

\usepackage[utf8]{inputenc}
\usepackage[margin=2.5cm]{geometry}
\usepackage{graphicx}
\usepackage{booktabs}
\usepackage{hyperref}
\usepackage{siunitx}
\usepackage{xcolor}
\usepackage{longtable}
\usepackage{multirow}
\usepackage{array}
\usepackage{float}        % Better float positioning
\usepackage{placeins}     % \FloatBarrier command
\usepackage{enumitem}     % Better list formatting

% For placeholder notes during drafting
\newcommand{\todo}[1]{\textcolor{red}{[TODO: #1]}}
\newcommand{\placeholder}[1]{\textcolor{blue}{#1}}

% Consistent table column types
\newcolumntype{L}[1]{>{\raggedright\arraybackslash}p{#1}}
\newcolumntype{C}[1]{>{\centering\arraybackslash}p{#1}}
\newcolumntype{R}[1]{>{\raggedleft\arraybackslash}p{#1}}

\journal{Data in Brief}

\begin{document}

\begin{frontmatter}

  \title{SOAR: A Scalable, Open, Automated, and Reproducible Urban Data Model for the EU}

  \author[bench]{Gareth Simons\corref{cor1}}
  \ead{g.simons@benchmarkurbanism.com}
  \cortext[cor1]{Corresponding author}

  \author[bench]{Second Author}
  \author[bench]{Third Author}

  \address[bench]{Benchmark Urbanism, \todo{Full postal address}, Country}

  \begin{abstract}
    Constructing spatial analytic datasets for urban planning is resource-intensive and data-source specific, limiting cross-context application. We present SOAR (Scalable, Open, Automated, and Reproducible), a pan-European urban data model providing morphology and accessibility metrics for 699 urban centres. SOAR combines EU-specific datasets---Eurostat High Density Clusters for boundary definition, Census 2021 for demographics, and Copernicus Urban Atlas for land cover---with Overture Maps for street networks, points of interest, and building footprints. The dataset provides over 100 metrics per street network node at multiple spatial scales (100--4,800m), encompassing network centrality, land-use accessibility and diversity, building morphology, green space proximity, and demographics. Processing is automated via open-source Python code with fixed parameters for reproducibility. Data are distributed as GeoPackage files in EPSG:3035 projection. This is a derived dataset from open sources processed at continental scale; it is not a curated or validated dataset, and users should assess data quality for their specific applications---particularly for point of interest data, which exhibits geographic variation in completeness. The primary contribution is reducing the barrier to pan-European urban comparison by providing pre-computed metrics in a consistent format.
  \end{abstract}

  \begin{keyword}
    urban morphology \sep
    accessibility \sep
    European cities \sep
    walkability \sep
    open data \sep
    reproducibility \sep
    street networks
  \end{keyword}

\end{frontmatter}

%% ============================================================================
%% INTRODUCTION
%% ============================================================================
\section{Introduction}

Constructing spatial analytic datasets for urban planning requires substantial data engineering: network analysis, proximity metrics, census integration, and land-use classification. This work is typically repeated for each study area. Existing large-scale datasets include 27,000 US street networks~[1], 931 UK towns~[2], and 50 cities across 29 countries~[3], but pan-European coverage with consistent methodology has been limited.

The TWIN2EXPAND consortium produced SOAR to address this gap. Continental scope requires broad-coverage sources (Overture Maps), while EU boundaries enable use of harmonised regional datasets:

\begin{itemize}
  \item Eurostat 2021 Urban Centres / High Density Clusters data to rigorously define urban extents, from which we have extracted 699 towns and cities.
  \item Eurostat 2021 homogenised 1$\times$1~km census statistics for population densities, employment levels, place of birth, and population change.
  \item Copernicus Urban Atlas data from which we derive urban blocks, building heights, and tree canopy coverage.
\end{itemize}

We use Overture Maps rather than OpenStreetMap for street networks, infrastructure, points of interest, and building footprints. Overture offers formalised release cycles, improved POI validation, and expanded building coverage via supplementary Google and Microsoft data~[4]. Network processing employs automated cleaning via cityseer~[5], including level-aware processing at bridges and classification-based edge merging.

SOAR (Scalable, Open, Automated, and Reproducible) provides pre-computed urban metrics for pan-EU comparison. The processing pipeline is open-source and reproducible. The primary contribution is the operational scale: deriving consistent metrics across 699 urban centres from heterogeneous open data sources simplifies downstream comparative work that would otherwise require substantial data engineering effort.

%% ============================================================================
%% SPECIFICATIONS TABLE
%% ============================================================================
\section*{Specifications Table}

\begin{table}[!htbp]
  \centering
  \small
  \begin{tabular}{@{}L{3.2cm}L{8.3cm}@{}}
    \toprule
    \textbf{Subject} & Geography, Urban Studies, Geospatial Data Science \\
    \midrule
    \textbf{Specific subject area} & Urban morphology, pedestrian accessibility, land-use diversity \\
    \midrule
    \textbf{Type of data} & Processed geospatial vector data (GeoPackage) \\
    \midrule
    \textbf{Data collection} & Derived from publicly available authoritative sources: Overture Maps Foundation (2024), Copernicus Urban Atlas (2018), Eurostat Census Grid (2021), Copernicus Height Model (2012) \\
    \midrule
    \textbf{Data source location} & Pan-European: 699 urban centres across EU member states. Coordinate Reference System: EPSG:3035 (ETRS89-LAEA Europe) \\
    \midrule
    \textbf{Data accessibility} & Repository: Zenodo. Data identification number: \todo{DOI}. URL: \todo{URL} \\
    \midrule
    \textbf{Related research article} & \todo{None / or cite future CEUS paper} \\
    \bottomrule
  \end{tabular}
\end{table}

%% ============================================================================
%% VALUE OF THE DATA
%% ============================================================================
\section*{Value of the Data}

\begin{itemize}
  \item \textbf{Consistent metrics across 699 cities}: Metrics are computed using identical methods and parameters, avoiding methodological inconsistencies that arise when combining city-specific datasets.

  \item \textbf{Multi-scale design}: Centrality metrics at 400m, 800m, 1,200m, 1,600m, and 4,800m (5--60 minute walking catchments). Accessibility metrics at 200m, 400m, 800m, 1,200m, and 1,600m.

  \item \textbf{Network-based aggregation}: Metrics computed along pedestrian routes rather than Euclidean buffers. A park 200m as-the-crow-flies but 800m by foot (due to barriers) is measured at 800m.

  \item \textbf{Distance-weighted accessibility}: Exponential decay ($e^{-\beta d}$) weights nearby POIs more heavily than distant ones.

  \item \textbf{Fine-grained resolution}: 80m street segments preserve within-neighbourhood heterogeneity.

  \item \textbf{Open and reproducible}: Processing code and parameters are documented; researchers can regenerate or extend the dataset.
\end{itemize}

%% ============================================================================
%% DATA DESCRIPTION
%% ============================================================================
\section{Data Description}

\subsection{Dataset Structure}

The dataset comprises 699 GeoPackage files, one per urban centre, following the naming convention \texttt{metrics\_\{bounds\_fid\}.gpkg} where \texttt{bounds\_fid} corresponds to the unique identifier from the source high-density cluster boundaries. Each GeoPackage contains three layers:

\begin{enumerate}
  \item \textbf{streets}: Street network nodes (80m segments) with computed metrics
  \item \textbf{buildings}: Individual building footprints with morphological attributes
  \item \textbf{blocks}: Urban blocks with aggregated statistics
\end{enumerate}

\todo{Insert Figure 1: Map of European coverage showing all 699 urban centres}

\subsection{Streets Layer Schema}

Table~\ref{tab:streets_schema} summarises the key attributes in the streets layer. Metrics are computed at multiple distance thresholds, indicated by the suffix (e.g., \texttt{\_400}, \texttt{\_800}). The complete schema with all attributes is provided in Supplementary Material (Section~S1).

\begin{table}[!htbp]
  \centering
  \caption{Summary of streets layer attributes (selected metrics shown; full schema in Supplementary Material Section~S1).}
  \label{tab:streets_schema}
  \small
  \begin{tabular}{@{}L{3.2cm}L{8.1cm}@{}}
    \toprule
    \textbf{Attribute group} & \textbf{Description} \\
    \midrule
    \texttt{cc\_beta\_*} & Closeness centrality (gravity-weighted) at distance thresholds \\
    \midrule
    \texttt{cc\_\{landuse\}\_*\_nw} & Unweighted POI count for land-use category \\
    \midrule
    \texttt{cc\_\{landuse\}\_*\_wt} & Distance-weighted POI count for land-use category \\
    \midrule
    \texttt{cc\_\{landuse\}\_nearest\_*} & Distance to nearest POI in category \\
    \midrule
    \texttt{cc\_hill\_*} & Land-use diversity (Hill numbers $q=0,1,2$) \\
    \midrule
    \texttt{cc\_green\_nearest\_*} & Distance to nearest green space \\
    \midrule
    \texttt{cc\_trees\_nearest\_*} & Distance to nearest tree canopy \\
    \midrule
    \texttt{cc\_*\_median\_*} & Morphology metrics (median aggregation) \\
    \midrule
    \texttt{density}, \texttt{t}, \texttt{emp} & Interpolated census metrics \\
    \bottomrule
  \end{tabular}
\end{table}

\subsection{Land-Use Categories}

Points of interest (POIs) from Overture Maps were classified into 11 aggregated categories for accessibility analysis (Table~\ref{tab:landuse}). The complete mapping from Overture's 2,000+ categories to analytical classes is detailed in Supplementary Material (Section~S2).

\begin{table}[!htbp]
  \centering
  \caption{Aggregated land-use categories derived from Overture Maps POI classification.}
  \label{tab:landuse}
  \small
  \begin{tabular}{@{}L{3.5cm}L{8cm}@{}}
    \toprule
    \textbf{Category} & \textbf{Includes} \\
    \midrule
    \texttt{eat\_and\_drink} & Restaurants, cafés, bars \\
    \texttt{retail} & Shops, markets, stores \\
    \texttt{business\_services} & Offices, professional services \\
    \texttt{public\_services} & Government, civic facilities \\
    \texttt{health\_medical} & Clinics, pharmacies, hospitals \\
    \texttt{education} & Schools, universities, libraries \\
    \texttt{accommodation} & Hotels, hostels \\
    \texttt{active\_life} & Sports, fitness, recreation \\
    \texttt{arts\_and\_entertainment} & Cinemas, theatres, venues \\
    \texttt{attractions\_and\_activities} & Museums, landmarks, parks \\
    \texttt{religious} & Places of worship \\
    \bottomrule
  \end{tabular}
\end{table}

\subsection{Buildings and Blocks Layers}

The buildings layer contains individual building footprints with morphological metrics including area, perimeter, compactness, orientation, estimated height (sampled from raster), and floor area ratio. The blocks layer provides urban block geometries derived from Urban Atlas with aggregated building coverage ratios and block-level morphology.

\todo{Insert Figure 2: Sample visualisation of metrics for one city (e.g., Barcelona or Vienna)}

%% ============================================================================
%% EXPERIMENTAL DESIGN, MATERIALS AND METHODS
%% ============================================================================
\section{Experimental Design, Materials and Methods}

\subsection{Study Area Definition}

Urban boundaries derive from the Eurostat High Density Clusters (HDENS-CLST) 2021 raster~[6]---a 1$\times$1~km grid identifying contiguous cells with $\geq$1,500 inhabitants/km\textsuperscript{2} and cumulative population $\geq$50,000. Vectorisation, filtering to continental Europe (EPSG:3035), and UK exclusion yielded 699 urban centres. City names were assigned via spatial join with Overture Maps administrative divisions.

\subsection{Data Sources}

Table~\ref{tab:sources} summarises input datasets. Street networks from Overture Maps~[4] were cleaned via cityseer~[5]: disconnected components removed, degree-2 nodes consolidated, edges decomposed to 80m segments ($\approx$1-minute walk). Level-aware processing prevents incorrect merging at bridges. POI categories from Overture's ``places'' theme were mapped to 11 analytical classes (Table~\ref{tab:landuse}). Land cover and tree canopy derive from Copernicus Urban Atlas~[7] and Street Tree Layer~[8] respectively. Building heights were sampled from the Digital Height Model~[9]. Demographics from Eurostat Census Grid 2021~[10] include population, employment, age structure, nationality, and migration at 1~km\textsuperscript{2} resolution.

\begin{table}[!htbp]
  \centering
  \caption{Input data sources and their roles in the processing pipeline.}
  \label{tab:sources}
  \small
  \begin{tabular}{@{}L{2.8cm}L{2.5cm}L{5.5cm}@{}}
    \toprule
    \textbf{Dataset} & \textbf{Source} & \textbf{Use} \\
    \midrule
    Street networks & Overture Maps & Centrality, accessibility \\
    Points of interest & Overture Maps & Land-use metrics \\
    Building footprints & Overture Maps & Morphology \\
    Urban Atlas 2018 & Copernicus & Blocks, green space \\
    Street Tree Layer & Copernicus & Tree proximity \\
    Height Model 2012 & Copernicus & Building heights \\
    Census Grid 2021 & Eurostat & Demographics \\
    \bottomrule
  \end{tabular}
\end{table}

\subsection{Data Processing Pipeline}

The pipeline comprises six stages: (1) boundary vectorisation from HDENS-CLST raster; (2--4) clipping of Copernicus datasets (Urban Atlas, Street Tree Layer, Height Model) to boundaries; (5) Overture data extraction per boundary; and (6) metric computation. All scripts support idempotent execution---existing outputs are skipped unless \texttt{--overwrite} is specified---enabling incremental processing and failure recovery.

To ensure accurate metric computation at boundary edges, input datasets are buffered beyond city boundaries. Street networks are buffered by 10km, ensuring that nodes near boundaries have complete network context for centrality calculations. Points of interest, buildings, and infrastructure are buffered by 2km, capturing all reachable amenities for boundary-proximate locations. Copernicus land cover and tree canopy data are clipped to boundary bounding boxes to ensure coverage for green space proximity metrics.

\subsection{Metric Computation}

\subsubsection{Network Centrality}
Network centrality metrics quantify a location's prominence within the street network and were computed using the cityseer package~[5]. The segment-based approach computes centrality relative to reachable street segments within distance thresholds, avoiding distortions from irregular node distributions. Beta-weighted (gravity index) closeness applies an explicit decay parameter~$\beta$:
\begin{equation}
  C_i^{\beta} = \sum_{j \neq i} w_j \cdot e^{-\beta \cdot d_{ij}}
\end{equation}
where $d_{ij}$ is network distance, $w_j$ is segment length, and $\beta$ controls distance decay~[5]. Harmonic closeness---appropriate for localised implementations constrained by threshold $d_{\max}$---was also computed. Metrics were generated at thresholds of 400m, 800m, 1,200m, 1,600m, and 4,800m, corresponding to 5, 10, 15, 20, and 60-minute walking catchments at 80m/min average pedestrian speed. The shorter thresholds (5--20 minutes) capture pedestrian-scale accessibility relevant to daily errands and commuting choices, while the 60-minute threshold characterises district-scale network structure.

\subsubsection{Land-Use Accessibility}
Accessibility metrics aggregate POI counts over the street network from identical locations as centrality computations, enabling direct correlation~[5]. Unlike Euclidean buffer approaches or zonal aggregation (which assign uniform values to all locations within administrative units), network-based accessibility follows actual pedestrian routes and varies continuously across street segments. For each land-use category $k$, both unweighted counts (number of reachable POIs within $d_{\max}$) and distance-weighted counts (applying exponential decay) were computed, alongside nearest-distance measures. Distance-weighted counts better reflect perceived accessibility: a café at 100m contributes more than one at 1,000m, capturing the empirical finding that amenity usage decays with distance.

Mixed-use diversity was quantified using Hill numbers~[5], the preferred diversity index because it adheres to the replication principle and uses units of effective species:
\begin{equation}
  {}^{q}D = \left( \sum_{k=1}^{K} p_k^q \right)^{1/(1-q)}
\end{equation}
where $p_k$ is the proportion of POIs in category $k$ and $q$ controls sensitivity to rare categories. At $q=0$, Hill numbers reduce to a simple count of distinct land-use types (species richness); at $q=1$, they approximate the exponential of Shannon entropy; at $q=2$, they approximate the inverse Simpson index, emphasising balance over richness. Following cityseer recommendations, we compute $q=0$, $q=1$, and $q=2$ variants in both unweighted and distance-weighted forms.

\subsubsection{Morphology}
Building morphology metrics were computed using standard geometric formulae: compactness as $4\pi A / P^2$, orientation via minimum bounding rectangle, and form factor as $A / (h \cdot P)$ where $h$ is height.

\subsubsection{Green Space Proximity}
Distance to nearest green space polygon edge was computed for each network node using spatial indexing.

\subsubsection{Demographic Interpolation}
Census grid values were interpolated to network nodes using linear interpolation from grid cell centroids.

\subsection{Implementation}

The pipeline is implemented in Python 3.12 with four modules: data ingestion (\texttt{src.data}), metric processing (\texttt{src.processing}), utilities (\texttt{src.tools}), and land-use classification (\texttt{src.landuse\_categories}).

Data loading abstracts heterogeneous sources through a unified interface. Network loading retrieves Overture ``connector'' and ``segment'' themes via STAC, transforms to EPSG:3035, constructs a NetworkX MultiGraph with node deduplication and edge attribute preservation, then applies cityseer cleaning (8m tolerance, 100-node component threshold). Building and infrastructure loading extract footprints and point features respectively. POI loading maps Overture's 2,000+ categories to 23 intermediate classes via CSV lookup, then to 11 analytical categories.

Land-use classification consolidates Overture's taxonomy in two stages: schema filtering retains 23 intermediate classes, then category merging yields 11 analytical categories (Table~\ref{tab:landuse}) aligned with established urban taxonomies.

The processing module (\texttt{generate\_metrics.py}) orchestrates metric computation across all boundaries. Networks undergo dual graph transformation (streets as nodes, intersections as edges) after 80m decomposition, enabling segment-level analysis.

\textbf{Centrality}: Beta-weighted (gravity index) and harmonic closeness at five thresholds (400m, 800m, 1,200m, 1,600m, 4,800m) via cityseer~[5].

**Accessibility**: POI counts (unweighted and distance-weighted) per land-use category within pedestrian-scale thresholds (200m, 400m, 800m, 1,600m); Hill number diversity indices ($q=0,1,2$) quantifying land-use heterogeneity~[5].

\textbf{Morphology}: Building metrics (area, compactness, orientation, height, volume, fractal dimension, shared walls) and block metrics (coverage ratio, shape) computed via momepy, then aggregated to network nodes at 100m and 200m thresholds using median and MAD statistics.

\textbf{Green space}: Network-distance proximity to Urban Atlas green classes and Street Tree Layer polygons at 1,600m threshold, using point-sampled boundaries at 20m intervals. Green and tree canopy areas are aggregated at 200m, 400m, and 800m thresholds.

\textbf{Demographics}: Census grid values interpolated to nodes via linear interpolation from cell centroids.

\subsection{Output Format}

Results are exported as GeoPackage files (\texttt{metrics\_\{bounds\_fid\}.gpkg}) in EPSG:3035, each containing three layers: \texttt{streets} (network nodes with all computed metrics), \texttt{buildings} (footprints with morphology), and \texttt{blocks} (polygons with coverage ratios). Dependencies are pinned via \texttt{pyproject.toml}; key packages include cityseer (network analysis), momepy (morphometrics), geopandas (spatial data), and overturemaps (data access). Processing parameters are fixed for reproducibility: 80m decomposition ($\approx$1-minute walk), 8m cleaning tolerance, centrality thresholds [400, 800, 1600, 4800]m (5--60 minutes), accessibility thresholds [200, 400, 800, 1600]m, morphology aggregation thresholds [100, 200]m.

\subsection{Data Quality Considerations}

SOAR is a derived dataset processed via automated pipeline from open data sources. It is not a curated or validated dataset. Users should assess fitness for purpose before analysis:

\begin{itemize}[nosep]
  \item \textbf{Point of interest data}: POI completeness varies geographically. Some Central and Western European cities (Germany, Netherlands) may exhibit high saturation coverage, while peripheral regions (parts of Spain, Romania, Bulgaria, southern Italy) may show systematic undersaturation. A companion paper~\todo{cite demonstrators} provides a multi-scale regression approach for assessing POI saturation per city.
  \item \textbf{Building heights}: Derived from 2012 Copernicus raster data; heights may be outdated in areas with recent construction.
  \item \textbf{Street networks}: Automated cleaning removes disconnected components and simplifies topology; some local idiosyncrasies or redundancies may persist.
  \item \textbf{Census interpolation}: Demographics are interpolated from 1~km\textsuperscript{2} grid cells to network nodes; this smooths within-cell variation.
\end{itemize}

The dataset is suitable for comparative analysis across cities at aggregate scales. Fine-grained analysis (e.g., individual street segments in undersaturated cities) may require additional validation against local data sources.

\subsection{Example Use Cases}

The following illustrate ways researchers might use SOAR's pre-computed metrics. These are not exhaustive, nor are the boundaries between them rigid. A companion paper~\todo{cite demonstrators paper} provides worked examples.

\begin{enumerate}[nosep]
  \item \textbf{Data quality filtering}: POI completeness varies geographically, with some regions exhibiting systematic undersaturation. This demonstrator applies multi-scale regression of POI counts against population densities to identify cities where crowdsourced data may be too sparse for reliable analysis, allowing researchers to filter or weight observations accordingly.

  \item \textbf{Multi-scale analysis}: Relationships observed within cities may differ from those observed between cities. SOAR enables within-city analysis while city-level aggregations support cross-city comparison, and the two perspectives can yield different---sometimes opposing---conclusions. This demonstrator explores the topic in the context of access to green spaces.

  \item \textbf{Access gap identification}: Distance-to-nearest metrics can reveal locations where distances to amenities or services are greater than average or exceed targeted thresholds. This demonstrator explores where distances to education and transport are greater than typical, helping to highlight areas of potential disadvantage warranting further investigation.

  \item \textbf{Predictive modelling}: Large-scale datasets from multiple cities can enable training of generalisable models. This demonstrator uses network centrality and population density to predict levels of eating and drinking establishments as well as business and services intensities. The consistent feature set across cities can support transfer learning or pooled models.

  \item \textbf{Benchmarking}: Cities can be ranked on standardised metrics enabling comparative assessment against peer cities or policy targets. This demonstrator ranks cities by walkable access to amenities and services across major land-use categories within pedestrian-scale distances.

  \item \textbf{Typology classification}: Clustering algorithms applied to street-level features can identify recurring neighbourhood types that transcend administrative boundaries, revealing morphological similarities across different urban contexts. This demonstrator applies clustering to identify urban morphological forms.

  \item \textbf{Site selection}: Large-scale datasets can be filtered by multiple criteria to identify candidate locations for new facilities, housing, or infrastructure investments. This demonstrator identifies locations with high centrality, mixed uses, and transport access, but lower population density as potential candidates for development.
\end{enumerate}

%% ============================================================================
%% ETHICS AND DECLARATION
%% ============================================================================
\section*{Ethics Statement}

This research did not involve human subjects, animal experiments, or data collected from social media platforms. All source datasets are publicly available under open licenses.

\section*{CRediT Author Statement}

\textbf{Gareth Simons}: Conceptualization, Methodology, Software, Data curation, Writing -- original draft, Visualization.
\todo{Add other authors and their contributions}

\section*{Declaration of Competing Interest}

The authors declare that they have no known competing financial interests or personal relationships that could have appeared to influence the work reported in this paper.

\section*{Acknowledgements}

\todo{Acknowledge TWIN2EXPAND project, funding bodies, and any individuals who assisted.}

This work was supported by \todo{the European Union's Horizon programme under grant agreement No. XXXXX (TWIN2EXPAND project)}.

\section*{Data Availability and Code Availability}

The dataset is available at Zenodo: \todo{Insert DOI and URL after deposit}.

\textbf{Supplementary Material} accompanies this article and includes:
\begin{itemize}
  \item Section S1: Complete streets layer schema with all $\sim$100 attributes
  \item Section S2: Full land-use classification mapping from Overture categories to analytical classes
  \item Section S3: Detailed metadata for all source datasets (HDENS-CLST, Overture Maps, Copernicus Urban Atlas, Street Tree Layer, Building Height, Census Grid)
  \item Section S4: Processing parameters (network cleaning, centrality computation, accessibility thresholds, POI saturation assessment)
  \item Section S5: Software dependencies with pinned versions
  \item Section S6: Computational requirements and parallelisation strategies
  \item Section S7: Data quality notes, known limitations, and QA procedures
  \item Section S8: Citation information for dataset, software, and source data
\end{itemize}

The processing workflow is implemented in the \textit{ebdptoolkit} repository (version 0.5.0), available at \url{https://github.com/UCL/ba-ebdp-toolkit}. The toolkit provides a complete, reproducible pipeline for generating urban metrics from open data sources (Overture Maps, Copernicus Urban Atlas, and census grids). The workflow comprises four stages: boundary generation from raster clusters, data ingestion from distributed sources, metric computation via network and morphological analysis, and POI saturation assessment via multi-scale regression.

To use the workflow: (1) clone the repository and install dependencies using \texttt{uv sync}; (2) download input datasets (boundaries raster, Overture dumps, Urban Atlas shapefiles) as documented in the README; (3) run \texttt{python -m src.processing.generate\_metrics} with parameters specifying output directory and boundary file; (4) optionally run \texttt{src/analysis/poi\_saturation\_notebook.py} to assess data completeness and generate quadrant classifications. All processing parameters (network decomposition, cleaning thresholds, distance scales) are configurable via function arguments. Complete documentation, data loading guidelines, and usage examples are provided in the repository's README and inline code comments. The toolkit is licensed under AGPL-3.0 and depends on open-source packages (cityseer, momepy, geopandas) ensuring full transparency and reproducibility across environments.

%% ============================================================================
%% REFERENCES
%% ============================================================================
\section*{References}

\begin{enumerate}
  \item G. Boeing, A multi-scale analysis of 27,000 urban street networks: Every US city, town, urbanized area, and Zillow neighborhood, Environ. Plan. B Urban Anal. City Sci. 47~(4) (2020) 590--608. \url{https://doi.org/10.1177/2399808318784595}

  \item G.D. Simons, Detection and prediction of urban archetypes at the pedestrian scale: computational toolsets, morphological metrics, and machine learning methods, Ph.D. thesis, UCL (University College London), 2021. \url{https://discovery.ucl.ac.uk/id/eprint/10134012/}

  \item W. Yap, F. Biljecki, A global feature-rich network dataset of cities and dashboard for comprehensive urban analyses, Sci. Data 10 (2023) 667. \url{https://doi.org/10.1038/s41597-023-02578-1}

  \item Overture Maps Foundation, Overture Maps Data -- 2024 Release [dataset], 2024. \url{https://overturemaps.org/}

  \item G. Simons, The cityseer Python package for pedestrian-scale network-based urban analysis, Environ. Plan. B Urban Anal. City Sci. 49~(9) (2022) 2356--2361. \url{https://doi.org/10.1177/23998083221133827}

  \item European Commission, Joint Research Centre, High Density Clusters -- HDENS-CLST 2021 [dataset], 2023. \url{https://ghsl.jrc.ec.europa.eu/}

  \item European Environment Agency, Urban Atlas 2018 [dataset], Copernicus Land Monitoring Service, 2020. \url{https://land.copernicus.eu/local/urban-atlas}

  \item European Environment Agency, Street Tree Layer 2018 [dataset], Copernicus Land Monitoring Service, 2020. \url{https://land.copernicus.eu/local/urban-atlas/street-tree-layer-stl-2018}

  \item European Environment Agency, Building Height 2012 [dataset], Copernicus Land Monitoring Service, 2020. \url{https://land.copernicus.eu/local/urban-atlas/building-height-2012}

  \item Eurostat, Census 2021 -- Population Grid [dataset], 2024. \url{https://ec.europa.eu/eurostat/web/gisco/geodata/population-distribution/geostat}
\end{enumerate}

\end{document}
